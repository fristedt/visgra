%%%%%%%%%%%%%%%%%%%%%%%%%%%%%%%%%%%%%%%%%
% Short Sectioned Assignment
% LaTeX Template
% Version 1.0 (5/5/12)
%
% This template has been downloaded from:
% http://www.LaTeXTemplates.com
%
% Original author:
% Frits Wenneker (http://www.howtotex.com)
%
% License:
% CC BY-NC-SA 3.0 (http://creativecommons.org/licenses/by-nc-sa/3.0/)
%
%%%%%%%%%%%%%%%%%%%%%%%%%%%%%%%%%%%%%%%%%

%----------------------------------------------------------------------------------------
%	PACKAGES AND OTHER DOCUMENT CONFIGURATIONS
%----------------------------------------------------------------------------------------

\documentclass[paper=a4, fontsize=11pt]{scrartcl} % A4 paper and 11pt font size

\usepackage[T1]{fontenc} % Use 8-bit encoding that has 256 glyphs
%\usepackage{fourier} % Use the Adobe Utopia font for the document - comment this line to return to the LaTeX default
\usepackage[english]{babel} % English language/hyphenation
\usepackage{amsmath,amsfonts,amsthm} % Math packages

\usepackage{lipsum} % Used for inserting dummy 'Lorem ipsum' text into the template
\usepackage{systeme} % Used for inserting dummy 'Lorem ipsum' text into the template

\usepackage{sectsty} % Allows customizing section commands
\allsectionsfont{\normalfont} % Make all sections centered, the default font and small caps

\usepackage{fancyhdr} % Custom headers and footers
\pagestyle{fancyplain} % Makes all pages in the document conform to the custom headers and footers
\fancyhead{} % No page header - if you want one, create it in the same way as the footers below
\fancyfoot[L]{} % Empty left footer
\fancyfoot[C]{} % Empty center footer
\fancyfoot[R]{\thepage} % Page numbering for right footer
\renewcommand{\headrulewidth}{0pt} % Remove header underlines
\renewcommand{\footrulewidth}{0pt} % Remove footer underlines
\setlength{\headheight}{13.6pt} % Customize the height of the header

\numberwithin{equation}{section} % Number equations within sections (i.e. 1.1, 1.2, 2.1, 2.2 instead of 1, 2, 3, 4)
\numberwithin{figure}{section} % Number figures within sections (i.e. 1.1, 1.2, 2.1, 2.2 instead of 1, 2, 3, 4)
\numberwithin{table}{section} % Number tables within sections (i.e. 1.1, 1.2, 2.1, 2.2 instead of 1, 2, 3, 4)

\setlength\parindent{0pt} % Removes all indentation from paragraphs - comment this line for an assignment with lots of text

\makeatletter
\newcommand{\Spvek}[2][r]{%
  \gdef\@VORNE{1}
  \left(\hskip-\arraycolsep%
    \begin{array}{#1}\vekSp@lten{#2}\end{array}%
  \hskip-\arraycolsep\right)}

\def\vekSp@lten#1{\xvekSp@lten#1;vekL@stLine;}
\def\vekL@stLine{vekL@stLine}
\def\xvekSp@lten#1;{\def\temp{#1}%
  \ifx\temp\vekL@stLine
  \else
    \ifnum\@VORNE=1\gdef\@VORNE{0}
    \else\@arraycr\fi%
    #1%
    \expandafter\xvekSp@lten
  \fi}
\makeatother

%----------------------------------------------------------------------------------------
%	TITLE SECTION
%----------------------------------------------------------------------------------------

\newcommand{\horrule}[1]{\rule{\linewidth}{#1}} % Create horizontal rule command with 1 argument of height

\title{	
\normalfont \normalsize 
\textsc{Introduction to Visualization and Computer Graphics, Fall 2016} \\ [25pt] % Your university, school and/or department name(s)
\horrule{0.5pt} \\[0.4cm] % Thin top horizontal rule
\huge Assignment 1\\ % The assignment title
\horrule{2pt} \\[0.5cm] % Thick bottom horizontal rule
}

\author{Hampus Fristedt} % Your name

\date{\normalsize\today} % Today's date or a custom date

\begin{document}

\maketitle % Print the title

%----------------------------------------------------------------------------------------
%	PROBLEM 1
%----------------------------------------------------------------------------------------

\section*{Task 1.1: Vector Calculus}

\subsection*{(a)}

\begin{align*} 
\begin{split}
  v_1 &= \Spvek{1; 4; 3}\\
  \|v_1\| &= \sqrt{1^2 + 4^2 + 3^2} = \sqrt{26}\\
  \hat{v}_1 &= \frac{v_1}{\sqrt{26}} = \Spvek{\frac{1}{\sqrt{26}}; \frac{4}{\sqrt{26}}; \frac{3}{\sqrt{26}}}\\
\end{split}					
\end{align*}

\begin{align*} 
\begin{split}
  v_2 &= \Spvek{0; 0; 12}\\
  \|v_2\| &= \sqrt{0^2 + 0^2 + 12^2} = 12\\
  \hat{v}_2 &= \frac{v_2}{12} = \Spvek{0; 0; 1}\\
\end{split}					
\end{align*}

\begin{align*} 
\begin{split}
  v_3 &= \Spvek{-2; 0; 1}\\
  \|v_3\| &= \sqrt{(-2)^2 + 0^2 + 1^2} = \sqrt{5}\\
  \hat{v}_3 &= \frac{v_3}{\sqrt{5}} = \Spvek{\frac{-2}{\sqrt{5}}; 0; \frac{1}{\sqrt{5}}}\\
\end{split}					
\end{align*}

\subsection*{(b)}

\[
  \Spvek{1; 4; 7}^T \Spvek{-2; 0; 3} = 1 \cdot -2 + 4 \cdot 0 + 7 \cdot 3 = 19
\]

\[
  \Spvek{4; 0; -5}^T \Spvek{0; 4; 0} = 4 \cdot 0 + 0 \cdot 4 + -5 \cdot 0 = 0
\]

\subsection*{(c)}

\[
  \Spvek{5; 3; 0} \times \Spvek{-2; 4; 0} = \Spvek{3 \cdot 0 - 0 \cdot 4; 0 \cdot -2 - 5 \cdot 0; 5 \cdot 4 - 3 \cdot -2} = \Spvek{0; 0; 26}
\]

\[
  \Spvek{1; 0; 0} \times \Spvek{0; 1; 1} = \Spvek{0 \cdot 1 - 0 \cdot 1; 0 \cdot 0 - 1 \cdot 1; 1 \cdot 1 - 0 \cdot 0} = \Spvek{0; -1; 1}
\]

%------------------------------------------------

\section*{Task 1.2: Cross Product}

\[
  a = \Spvek{a_1; a_2; a_3}
  b = \Spvek{b_1; b_2; b_3}
  c = a \times b = \Spvek{a_2b_3 - a_3b_2; a_3b_1 - a_1b_3; a_1b_2 - a_2b_1}
\]

\begin{align*} 
\begin{split}
  a \cdot c &= a_1(a_2b_3 - a_3b_2) + a_2(a_3b_1 - a_1b_3) + a_3(a_1b_2 - a_2b_1)\\
	    &= a_1a_2b_3 - a_1a_3b_2 + a_2a_3b_1 - a_1a_2b_3 + a_1a_3b_2 - a_2a_3b_1\\
     &= 0
\end{split}
\end{align*}

\begin{align*} 
\begin{split}
  b \cdot c &= b_1(a_2b_3 - a_3b_2) + b_2(a_3b_1 - a_1b_3) + b_3(a_1b_2 - a_2b_1)\\
	    &= a_2b_1b_3 - a_3b_1b_2 + a_3b_1b_2 - a_1b_2b_3 + a_1b_2b_3 - a_2b_1b_3\\
     &= 0
\end{split}
\end{align*}

Since both dot products are zero, the vectors are orthagonal by definition (two
vectors are orthagonal if their dot product is zero).

\section*{Task 1.3: Transformations}

\subsection*{(a)}

Reflection about a line through the origin which makes an angle $\theta$ with the
x-axis is represented by the matrix

\[
  \begin{bmatrix}
    \cos{2\theta} & \sin{2\theta} & 0 \\
    \sin{2\theta} & -\cos{2\theta} & 0 \\
    0 & 0 & 1
  \end{bmatrix}
\]

(https://en.wikipedia.org/wiki/Coordinate\_rotations\_and\_reflections)

We can use this to reflect over the line $y = -x$. With $\theta = -45^\circ$ we
obtain the reflection matrix

\[
  \begin{bmatrix}
    0 & -1 & 0 \\
    -1 & 0 & 0 \\
    0 & 0 & 1
  \end{bmatrix}
\]

A translation $T(3, 3)$ is then necessary to represent reflection over $y = -x +
3$. Our final reflection matrix is then

\[
  \begin{bmatrix}
    1 & 0 & 3 \\
    0 & 1 & 3 \\
    0 & 0 & 1
  \end{bmatrix}
  \begin{bmatrix}
    0 & -1 & 0 \\
    -1 & 0 & 0 \\
    0 & 0 & 1
  \end{bmatrix}
  =
  \begin{bmatrix}
    0 & -1 & 3 \\
    -1 & 0 & 3 \\
    0 & 0 & 1
  \end{bmatrix}
\]

\subsection*{(b)}

Let

\[
  T = 
  \begin{bmatrix}
    a & b & c \\
    x & y & z \\
    0 & 0 & 1
  \end{bmatrix}
\]

Substituting the three points $x_1$, $x_2$, $x_3$ and their images
$x^\prime_1$, $x^\prime_2$, $x^\prime_3$ into $x^\prime_i = Tx_i$ we get two
systems of linear equations, both with 3 variables and 3 equations:

\[
  \systeme*{b + c = -1, a + c = -1, a + 2b + c = 1}
  \Rightarrow
  \systeme*{a = 1, b = 1, c = -2}
\]
\[
  \systeme*{y + z = 5, x + z = 1, x + 2y + z = 5}
  \Rightarrow
  \systeme*{x = -2, y = 2, z = 3}
\]

The final transformation matrix is then

\[
  T =
  \begin{bmatrix}
    1 & 1 & -2 \\
    -2 & 2 & 3 \\
    0 & 0 & 1
  \end{bmatrix}
\]

\subsection*{(c)}

Let
\[
  T = 
  \begin{bmatrix}
    a & b & c \\
    d & e & f \\
    0 & 0 & 1
  \end{bmatrix}
\]

Then

\[
  T
  \begin{bmatrix}
    x \\
    y \\
    1
  \end{bmatrix}
  =
  \begin{bmatrix}
    a & b & c \\
    d & e & f \\
    0 & 0 & 1
  \end{bmatrix}
  \begin{bmatrix}
    x \\
    y \\
    1
  \end{bmatrix}
  =
  \begin{bmatrix}
    \cos^2\alpha x - \sin^2\alpha y + \frac{\pi}{2}\\
    \tan^2\alpha y - \cos^2\alpha x + \pi\\
    1
  \end{bmatrix}
\]
and
\[
  \systeme*{a = \cos^2\alpha, b = -\sin^2\alpha, c = \frac{pi}{2}, d = \tan^2\alpha, e = -\cos^2\alpha, f = \pi}
\]

The final transformation matrix is

\[
  T = 
  \begin{bmatrix}
  \cos^2\alpha & -\sin^2\alpha & \frac{pi}{2}\\
  \tan^2\alpha & -\cos^2\alpha &  \pi\\
    0 & 0 & 1
  \end{bmatrix}
\]

\subsection*{(d)}

Using the same technique as in (b) and (c) we get

\[
  T =
  \begin{bmatrix}
    0 & 0 & \pi \\

  \end{bmatrix}
\]

\section*{Task 1.4: Rectangle}

First we translate the rectangle so that its center is at the origin.

\[
  \begin{bmatrix}
    1 & 0 & -0.5\\
    0 & 1 & -0.5\\
    0 & 0 & 1
  \end{bmatrix}
\]

Then we scale by $\sqrt{(2\sqrt2)^2 + (2\sqrt2)^2} = 4$ in both $x$ and $y$

\[
  \begin{bmatrix}
    4 & 0 & 0\\
    0 & 4 & 0\\
    0 & 0 & 1
  \end{bmatrix}
\]

Then we rotate $45^\circ$

\[
  \begin{bmatrix}
    \cos{45^\circ} & -\sin{45^\circ} & 0\\
    \sin{45^\circ} & \cos{45^\circ} & 0\\
    0 & 0 & 1
  \end{bmatrix}
\]

Then we translate the rectangle so that its center is at $(10, 7+2\sqrt{2})$

\[
  \begin{bmatrix}
    1 & 0 & 10\\
    0 & 1 & 7 + 2\sqrt{2}\\
    0 & 0 & 1
  \end{bmatrix}
\]

The final transformation matrix T is

\[
  T=
  \begin{bmatrix}
    1 & 0 & 10\\
    0 & 1 & 7 + 2\sqrt{2}\\
    0 & 0 & 1
  \end{bmatrix}
  \begin{bmatrix}
    \cos{45^\circ} & -\sin{45^\circ} & 0\\
    \sin{45^\circ} & \cos{45^\circ} & 0\\
    0 & 0 & 1
  \end{bmatrix}
  \begin{bmatrix}
    4 & 0 & 0\\
    0 & 4 & 0\\
    0 & 0 & 1
  \end{bmatrix}
  \begin{bmatrix}
    1 & 0 & -0.5\\
    0 & 1 & -0.5\\
    0 & 0 & 1
  \end{bmatrix}
\]

\section*{Task 1.5: Rotations (Bonus)}

Using the trigonometric identities

\[
  \sin(\alpha \pm \beta )=\sin \alpha \cos \beta \pm \cos \alpha \sin \beta
\]
\[
  \cos(\alpha \pm \beta )=\cos \alpha \cos \beta \mp \sin \alpha \sin \beta
\]

we get

\begin{align*} 
\begin{split}
  R(\theta_1)R(\theta_2) &=
  \begin{bmatrix}
    \cos{\theta_1} & -\sin{\theta_1}\\
    \sin{\theta_1} & \cos{\theta_1}\\
  \end{bmatrix}
  \begin{bmatrix}
    \cos{\theta_2} & -\sin{\theta_2}\\
    \sin{\theta_2} & \cos{\theta_2}\\
  \end{bmatrix}\\
  &=
  \begin{bmatrix}
    \cos\theta_1\cos\theta_2 - \sin\theta_1\sin\theta_2 & -\cos\theta_1\sin\theta_2 - \sin\theta_1\cos\theta_2\\
    \sin\theta_1\cos\theta_2 + \cos\theta_1\sin\theta_2 & -\sin\theta_1\sin\theta_2 + \cos\theta_1\cos\theta_2
  \end{bmatrix}\\
  &=
  \begin{bmatrix}
    \cos{(\theta_1 + \theta_2)} & -\sin{(\theta_1 + \theta_2)}\\
    \sin{(\theta_1 + \theta_2)} & \cos{(\theta_1 + \theta_2)}
  \end{bmatrix}\\
  &= 
  R(\theta_1 + \theta_2)
\end{split}
\end{align*} 
QED.
\end{document}
