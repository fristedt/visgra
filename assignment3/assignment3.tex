%%%%%%%%%%%%%%%%%%%%%%%%%%%%%%%%%%%%%%%%%
% Short Sectioned Assignment
% LaTeX Template
% Version 1.0 (5/5/12)
%
% This template has been downloaded from:
% http://www.LaTeXTemplates.com
%
% Original author:
% Frits Wenneker (http://www.howtotex.com)
%
% License:
% CC BY-NC-SA 3.0 (http://creativecommons.org/licenses/by-nc-sa/3.0/)
%
%%%%%%%%%%%%%%%%%%%%%%%%%%%%%%%%%%%%%%%%%

%----------------------------------------------------------------------------------------
%	PACKAGES AND OTHER DOCUMENT CONFIGURATIONS
%----------------------------------------------------------------------------------------

\documentclass[paper=a4, fontsize=11pt]{scrartcl} % A4 paper and 11pt font size

\usepackage[T1]{fontenc} % Use 8-bit encoding that has 256 glyphs
%\usepackage{fourier} % Use the Adobe Utopia font for the document - comment this line to return to the LaTeX default
\usepackage[english]{babel} % English language/hyphenation
\usepackage{amsmath,amsfonts,amsthm} % Math packages

\usepackage{lipsum} % Used for inserting dummy 'Lorem ipsum' text into the template
\usepackage{systeme} % Used for inserting dummy 'Lorem ipsum' text into the template

\usepackage{sectsty} % Allows customizing section commands
\allsectionsfont{\normalfont} % Make all sections centered, the default font and small caps

\usepackage{fancyhdr} % Custom headers and footers
\pagestyle{fancyplain} % Makes all pages in the document conform to the custom headers and footers
\fancyhead{} % No page header - if you want one, create it in the same way as the footers below
\fancyfoot[L]{} % Empty left footer
\fancyfoot[C]{} % Empty center footer
\fancyfoot[R]{\thepage} % Page numbering for right footer
\renewcommand{\headrulewidth}{0pt} % Remove header underlines
\renewcommand{\footrulewidth}{0pt} % Remove footer underlines
\setlength{\headheight}{13.6pt} % Customize the height of the header

\numberwithin{equation}{section} % Number equations within sections (i.e. 1.1, 1.2, 2.1, 2.2 instead of 1, 2, 3, 4)
\numberwithin{figure}{section} % Number figures within sections (i.e. 1.1, 1.2, 2.1, 2.2 instead of 1, 2, 3, 4)
\numberwithin{table}{section} % Number tables within sections (i.e. 1.1, 1.2, 2.1, 2.2 instead of 1, 2, 3, 4)

\setlength\parindent{0pt} % Removes all indentation from paragraphs - comment this line for an assignment with lots of text

\makeatletter
\newcommand{\Spvek}[2][r]{%
  \gdef\@VORNE{1}
  \left(\hskip-\arraycolsep%
  \begin{array}{#1}\vekSp@lten{#2}\end{array}%
\hskip-\arraycolsep\right)}

\def\vekSp@lten#1{\xvekSp@lten#1;vekL@stLine;}
\def\vekL@stLine{vekL@stLine}
\def\xvekSp@lten#1;{\def\temp{#1}%
  \ifx\temp\vekL@stLine
  \else
  \ifnum\@VORNE=1\gdef\@VORNE{0}
  \else\@arraycr\fi%
  #1%
  \expandafter\xvekSp@lten
\fi}
\makeatother

%----------------------------------------------------------------------------------------
%	TITLE SECTION
%----------------------------------------------------------------------------------------

\newcommand{\horrule}[1]{\rule{\linewidth}{#1}} % Create horizontal rule command with 1 argument of height

\title{	
  \normalfont \normalsize 
  \textsc{Introduction to Visualization and Computer Graphics, Fall 2016} \\ [25pt] % Your university, school and/or department name(s)
  \horrule{0.5pt} \\[0.4cm] % Thin top horizontal rule
  \huge Assignment 3\\ % The assignment title
  \horrule{2pt} \\[0.5cm] % Thick bottom horizontal rule
}

\author{Hampus Fristedt} % Your name

\date{\normalsize\today} % Today's date or a custom date

\begin{document}

\maketitle % Print the title

%----------------------------------------------------------------------------------------
%	PROBLEM 1
%----------------------------------------------------------------------------------------

\section*{Task 3.1: Linear Index Mapping}

\subsection*{(a)}

Let $f: (i, j, k) \rightarrow l$. Then $f(i, j, k) = i + n_xj + n_xn_yk$.

\subsection*{(b)}

Let $g: l \rightarrow (i, j, k)$. Then $g(l) = (l \mod n_x, \left \lfloor\frac{l}{n_x} \right \rfloor \mod n_y, \left \lfloor\frac{l}{n_xn_y} \right \rfloor)$.

\section*{Task 3.2: Order Independence of Bi-linear Interpolation}

Let $f_x(x, y)$ denote the bi-linear interpolation starting with the x-direction and $f_y(x, y)$ denote the bi-linear interpolation starting with the y-direction.

\subsubsection*{x-direction first}

We start by applying linear interpolating on the pairs $(f_{00}, f_{10})$ and $(f_{01}, f_{11})$ to obtain the values $f(x, 0)$ and $f(x, 1)$.

\begin{align*}
  \begin{split}
    f(x, 0) &= (1 - x)f_{00} + xf_{10}\\
    f(x, 1) &= (1 - x)f_{01} + xf_{11}
  \end{split}
\end{align*}

We then interpolate between those values to obtain $f_x(x, y)$.

\begin{align*}
  \begin{split}
    f_x(x, y) &= (1 - y)f(x, 0) + yf(x, 1)\\
		   &= (1 - y)(1 - x)f_{00} + (1 - y)xf_{10} + y(1 - x)f_{01} + xyf_{11}\\
    &= (1-x)(1-y)f_{00} + x(1 - y)f_{10} + y(1 - x)f_{01} + xyf_{11}
  \end{split}
\end{align*}

\subsubsection*{y-direction first}

If we instead start by interpolating between $(f_{00}, f_{01})$ and $(f_{10}, f_{11})$ we get the following

\begin{align*}
  \begin{split}
    f(0, y) &= (1 - y)f_{00} + yf_{01}\\
    f(1, y) &= (1 - y)f_{10} + yf_{11}
  \end{split}
\end{align*}

and after interpolating between those values in the x-direction we get

\begin{align*}
  \begin{split}
    f_y(x, y) &= (1 - x)f(0, y) + yf(1, y)\\
	    &= (1 - x)(1 - y)f_{00} + (1 - x)yf_{01} + x(1 - y)f_{10} + xyf_{11}\\
     &= (1-x)(1-y)f_{00} + x(1 - y)f_{10} + y(1 - x)f_{01} + xyf_{11}\\
    &= f_x(x, y)
  \end{split}
\end{align*}

$\hfill\square$

\section*{Task 3.3: Interpolation}

\subsection*{(a)}

Let $s(x, y) = ax + by + c$. We can use the provided scalar values to get a
linear system of equations.

\begin{align*}
  \begin{split}
    s(0, 0) &= c = 1 \Rightarrow c = 1\\
    s(1, 0) &= a + c = 2 \Rightarrow a = 1\\
    s(0, 1) &= b + c = 3 \Rightarrow b = 2
  \end{split}
\end{align*}

Thus, $s(x, y) = x + 2y + 1$.

\subsection*{(b)}

\begin{align*}
  \begin{split}
    s(1/2, 0) &= 1/2 + 1 = 3/2\\
    s(0, 1/2) &= 2 \cdot 1/2 + 1 = 2\\
    s(1/2, 1/2) &= 1/2 + 2 \cdot 1/2 + 1 = 5/2\\
    s(2/3, 2/3) &= 2/3 + 2 \cdot 2/3 + 1 = 3
  \end{split}
\end{align*}

\section*{Task 3.4: Triangle Basis Function Gradients}
\end{document}
